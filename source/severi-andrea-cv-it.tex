\documentclass[]{resume-openfont}

\pagestyle{fancy}
\resetHeaderAndFooter

%--------------------------------------------------------------
% Convenience command - make it easy to fill template

% Create job position command. Parameters: company, position, location, when
\newcommand{\resumeHeading}[4]{\runsubsection{\uppercase{#1}}\descript{ | #2}\hfill\location{#3 | #4}\fakeNewLine}

% Create education heading. Parameters: Name, degree, location, when
\newcommand{\educationHeading}[4]{\runsubsection{#1}\hspace*{\fill}  \location{#3 | #4}\\
\descript{#2}\fakeNewLine}

% Create project heading. Parameters: Name, link, Tech stack
\newcommand{\projectHeading}[3]{\Project{#1}{#2}
\descript{#3}\\}

\newcommand{\projectHeadingWithDate}[4]{\Project{#1}{#2}
\descript{#3 | #4}\\}

% Parameters: courses
\newcommand{\courseWork}[1]{\textbf{Coursework:} #1}

% Parameters: courses
\newcommand{\teacherAssistant}[1]{\textbf{Teacher Assistant (TA):} #1}
 
%--------------------------------------------------------------
\begin{document}

%--------------------------------------------------------------
%     Profile
%--------------------------------------------------------------
\newcommand{\yourName}{Andrea Severi}
% How you want it to show up on the resume
\newcommand{\yourWebsite}{seve-andre.github.io/}
% vs how you want it to show up. If it's the same you can just replace "\yourWebsiteLink" with "yourWebsite"
\newcommand{\yourWebsiteLink}{https://seve-andre.github.io/}
\newcommand{\myLocation}{Cesena, IT }
\newcommand{\myPhoneNumber}{342-3209-388}
\newcommand{\yourEmail}{andrea.severi.dev@gmail.com}
\newcommand{\githubUserName}{seve-andre}
\newcommand{\linkedInUserName}{andrea-severi}

% An alternate profile section 
% \alignProfileTable
% \begin{tabular*}{\textwidth}{l@{\extracolsep{\fill}}r}
%     \ralewayBold{\href{\yourWebsiteLink}{\Large \yourName}} & 
%     Email : \href{mailto:\yourEmail}{\yourEmail}
%     \\
%     \href{https://github.com/\githubUserName}{GitHub://\githubUserName} & 
%     Mobile : \myPhoneNumber
%     \\
%     \href{https://www.linkedin.com/in/\linkedInUserName}{LinkedIn://\linkedInUserName} & Website : \href{\yourWebsiteLink}{\yourWebsite}
%     \\
% \end{tabular*}

\begin{center}
    \Huge \scshape \latoRegular{\yourName} \\ \vspace{7pt}
    \normalsize \href{mailto:\yourEmail}{\underline{\yourEmail}} $|$
    \href{https://www.linkedin.com/in/andrea-severi/}{\underline{linkedIn/\linkedInUserName}} $|$
    \href{https://github.com/seve-andre}{\underline{github/\githubUserName}} 
    $|$ \href{\yourWebsiteLink}{\underline{\yourWebsite}}
    \\ \vspace{3pt}
    \myLocation $|$ \myPhoneNumber
\end{center}

%--------------------------------------------------------------
%     Education
%--------------------------------------------------------------
\section{Formazione}
% Put school first and degree second if your school is reputable
\educationHeading{Laurea Triennale in Ingegneria e Scienze Informatiche}{Università di Bologna}{Cesena, IT}{Lug 2023}

%\teacherAssistant{World Wide Web Information Systems Development}
% \courseWork{Data Structures and Algorithms; Operating Systems;  Computer Security; Software Testing; Advanced Networking; Big Data Analytics}
\sectionsep

%--------------------------------------------------------------
%     Experience
%--------------------------------------------------------------
\section{Esperienze lavorative}
\resumeHeading{\href{https://flashstart.com/}{Flashstart}}{Tirocinante in Intelligenza Artificiale}{Cesena, IT}{Giu 2022 – Lug 2022}
\begin{bullets}
    \item Acquisizione di siti web appartenenti a categorie pre-esistenti
    \item Estrazione delle parti più rilevanti del testo dei siti web precedentemente menzionati e individuazione, per ciascuno, della lingua principale
    \item \textbf{Rimozione delle stopwords} mediante l'utilizzo della libreria NLTK in Python
    \item Applicazione di \textbf{stemming} e \textbf{lemmatization} al dataset utilizzando la libreria NLTK in Python
    \item Creazione di \textbf{wordclouds} per visualizzare la frequenza e l'importanza di ciascuna parola presente nel dataset
    \item Sviluppo di una \textbf{rete neurale} per la classificazione testuale mediante l'uso delle librerie Keras e TensorFlow
\end{bullets}
\sectionsep

\resumeHeading{\href{https://flashstart.com/}{Flashstart}}{Tirocinante in Intelligenza Artificiale}{Cesena, IT}{Lug 2021 - Ago 2021}
\begin{bullets}
    \item Acquisizione di dataset contenenti \textbf{parole straniere} per migliorare l'accuratezza della rete neurale esistente
    \item Rimozione delle \textbf{stopwords} mediante l'utilizzo della libreria NLTK in Python
    \item Applicazione di \textbf{stemming} e \textbf{lemmatization} al dataset utilizzando la libreria NLTK in Python
    \item Implementazione di un \textbf{traduttore} per tradurre i dataset in inglese
    \item Miglioramento della \textbf{qualità} dei dataset pre-esistenti
\end{bullets}
\sectionsep

%--------------------------------------------------------------
%     Projects
%--------------------------------------------------------------
\href{https://github.com/seve-andre?tab=repositories}{\section{Progetti \faExternalLink}}
% \section{Progetti}

\projectHeadingWithDate{smart garden}{https://github.com/seve-andre/smart-garden}{C++, IoT, Arduino, Java}{Lug 2022}
Sistema Embedded \& IoT che implementa una versione semplificata di un giardino smart, ovvero un sistema intelligente che monitora e controlla lo stato del giardino tramite sensori, protocollo HTTP, linea Seriale e Bluetooth. \\
\sectionsep
  
\projectHeadingWithDate{myunibo}{https://github.com/seve-andre/myUniBO}{Kotlin, Jetpack Compose, Android}{Mag - Giu 2022}
Rivisitazione dell'app ufficiale \href{https://play.google.com/store/apps/details?id=com.myunibo}{myUniBo} dell'Università di Bologna che permette agli studenti di prenotare esami, vedere le lezioni del giorno, i voti e le statistiche riguardanti la propria carriera universitaria in maniera più semplice e intuitiva.
\sectionsep

\projectHeadingWithDate{airline traffic simulator}{https://github.com/seve-andre/OOP20-alt-sim}{Java, JavaFX, MVC}{Mag - Ago 2021}
Videogioco desktop 2D sviluppato in gruppo e ispirato a \href{https://www.youtube.com/watch?v=KTH084KeFBc}{Flight Control}. L'obiettivo del giocatore è quello di far atterrare il maggior numero possibile di aerei, senza farli scontrare l'uno con l'altro. La difficoltà aumenta ad ogni atterraggio, il quale conferisce punti al giocatore.
\sectionsep

\projectHeadingWithDate{casine}{https://github.com/seve-andre/casine}{Tauri, Rust, SvelteKit, Typescript}{Feb-Mag 2023}
Applicazione desktop per la gestione di appartamenti in affitto che memorizza gli ospiti, i loro dati e le informazioni riguardanti case e appartamenti. Permette di calcolare il prezzo finale, in base al numero di ospiti e la loro età. \\
\sectionsep

% \projectHeading{!!! jetpack compose extended / java more}{https://github.com/Aarif123456/GoalOrientedBehaviour}{C\#}
% A capture-the-flag-styled shooting game composed of AI-controlled players. The agents change their goals based on various factors, such as their health, current weapon, their personality and what they see.\\
% \sectionsep

\projectHeadingWithDate{todoer}{https://github.com/seve-andre/todoer}{Kotlin, Jetpack Compose, Android}{Apr 2023}
Todoer è un'app Android molto utile per organizzare e gestire le attività quotidiane. Permette di creare liste di cose da fare, impostare scadenze e promemoria. \\
\sectionsep

%--------------------------------------------------------------
%     Skills
%--------------------------------------------------------------
\section{Competenze}
\begin{skillList}
    \singleItem{Linguaggi:}{Java, Kotlin, Python, Rust, TypeScript, SQL}
    \\
    \singleItem{Sviluppo Web:}{Svelte, Vue.js, PHP, JavaScript, TypeScript, HTML/CSS}
    \\
    \singleItem{Tecnologie:}{Git, MySQL, SQLite, LaTeX}
\end{skillList}
\end{document}